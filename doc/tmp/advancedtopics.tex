\chapter{Advanced Topics}
\label{advtopic}
\minitoc
\section{Operating System and Boot Image Details}
The operating system structure in the Retail Project is intended to provide
a system for the simple generation of operating system images for cash register
systems. An image is a data structure equipped with a file system that can
be linked to the main memory (RAM) or to a hard disk. The preferred software
package format for generating and adapting images is RPM. The option
to adapt an image manually exists, but should be avoided. Every form of
software that is used in this area in cooperation with SUSE LINUX AG is an
RPM package. The implementation is script-based and uses the Perl and Bash
languages.


\index{configuration files!config.<MAC Address>|(}
\subsection{The CR Configuration File}
\label{section:confmac}
This section describes the cash register configuration file:

\begin{Command}{8cm}
            config.$<$MAC Address$>$
\end{Command}

The configuration file contains data about
image, configuration, synchronization, or partition parameters.
The configuration file is loaded from the TFTP server directory \textit{/tftpboot/CR} via TFTP
for previously installed cash registers. New cash registers are immediately registered and a new
configuration file with the corresponding MAC address is created.

For SLRS system administrators there is no need to edit the CR configuration files manually.
All information is gathered from the LDAP entries of the administration server.
For testing and debugging purposes the \textit{RELOAD\_IMAGE} and \textit{RELOAD\_CONFIG} feature
is available.

Below, find an example of a cash register configuration file:
\begin{verbatim}
IMAGE=/dev/hda2;image/browser;1.1.1;192.168.1.1;4096
CONF=/CR/00:30:05:1D:75:D2/ntp.conf;/etc/ntp.conf;192.168.1.1;1024,     \
     /CR/00:30:05:1D:75:D2/XF86Config;/etc/X11/XF86Config;192.168.1.1;1024
PART=200;S;x,300;L;/,500;L;/opt,x;L;/home
DISK=/dev/hda
\end{verbatim}

The following format is used:

\begin{Command}{14cm}
            \textbf{IMAGE}=\underline{device;name;version;srvip;bsize;compressed},...,\\
            \textbf{SYNC}=syncfilename;srvip;bsize\\
            \textbf{CONF}=\underline{src;dest;srvip;bsize},...,
                          src;dest;srvip;bsize\\
            \textbf{PART}=\underline{size;id;Mount},...,size;id;Mount\\
			\textbf{JOURNAL}=ext3\\
            \textbf{DISK}=device
\end{Command}

\begin{itemize}
            \item \textbf{IMAGE}\\
                  Specifies which image(s) (name) should be loaded with which
                  version (version) and to which storage device (device) it
                  should be linked, e.g., \textbf{/dev/ram1} or
				  \textbf{/dev/hda2}. Multiple image downloads are possible but
                  the first listed image has to be the main operating system image
                  \footnote{~\textbf{Linux Newbies:}
				  The POS client partition (device)
				  \textbf{hda2} defines the root file system "/" and \texttt{hda1} is used
				   for the swap partition.
		 		  The numbering of the hard disk device should not be confused with
        		  the RAM disk device, where \texttt{/dev/ram0} is used for the initial
				  RAM disk and can not be used as storage device for the second stage
                  POS image. SUSE recommends to use the device \texttt{/dev/ram1} for the
                  RAM disk.}. If the hard drive is used, a corresponding partitioning must
                  be performed.
		    \item \textbf{srvip}\\
                  Specifies the server IP address for the TFTP download.
                  Must always be indicated, except in PART.
            \item \textbf{bsize}\\
                  Specifies the block size for the TFTP download. Must always
                  be indicated, except in PART. If the block size is too small
                  according to the maximum number of data packages (32768),
                  \textbf{linuxrc} will automatically calculate a new blocksize for
                  the download.
			\item \textbf{compressed}\\
                  Specifies if the image file on the TFTP server is compressed and
                  handles it accordingly. To specify a compressed image download only
                  the keyword \textbf{"'compressed"'} needs to be added. If compressed
                  is not specified the standard download workflow is used. \textbf{Note:}
                  The download will fail if you specify "'compressed"' and the image isn't
                  compressed. It will also fail if you don't specify "'compressed"'
                  but the image is compressed. The name of the compressed image has
                  to contain the suffix \textbf{.gz} and needs to be compressed with the
                  \textbf{gzip} tool.
		    \item \textbf{SYNC}\\
                  Contains the name of a file (syncfilename) downloaded
                  via TFTP whose contents indicate the number of seconds of
                  waiting time before the register image is downloaded.
            \item \textbf{CONF}\\
                  Specifies a comma-separated list of source:target
                  configuration files. The source (src) corresponds to the path
                  on the TFTP server and is loaded via TFTP. The
                  download is made to the file on the register
                  indicated by the target
                  (dest).
            \item \textbf{PART}\\
                  Specifies the partitioning data. The comma-separated list
                  must contain the size (size), the type number (id), and the
                  mount point (Mount).
                  \begin{itemize}
                  \item The first element of the list must define the swap
                        partition.
                  \item The second element of the list must define the
                        \textbf{root} partition.
                  \item The swap partition must not contain a mount point.
                        A lowercase letter \textbf{x} must be set instead.
                  \item If a partition should take all the space left on
                        a disk one can set a lower \textbf{x} letter as
                        size specification.
                  \end{itemize}
			\item \textbf{JOURNAL}\\
				  Specifies a journal to be added to the filesystem. The value
                  for this parameter must be set to \textbf{ext3} because the
                  only journaled filesystem we are using is ext3. the JOURNAL
				  parameter will be evaluated only if DISK has been set as well.
            \item \textbf{DISK}\\
                  Specifies the hard disk. Used only with PART and defines
                  the device via which the hard disk can be addressed,
                  e.g., \textbf{/dev/hda}.
                        \item \textbf{RELOAD\_IMAGE}\\
                  If set to a non-empty string, forces the configured
                  image to be loaded from the server even if the image on
                  the disk is up-to-date. Used mainly for debugging
                  purposes, this option only makes sense on diskful
                  systems. Using \texttt{posldap2crconfig.pl} will overwrite
				  this optional feature of the CR configuration file.
            \item \textbf{RELOAD\_CONFIG}\\
                  If set to an non-empty string, forces all config files
                  to be loaded from the server. Used mainly for debugging
                  purposes, this option only makes sense on diskful
                  systems.
				  Using \texttt{posldap2crconfig.pl} will overwrite
				  this optional feature of the CR configuration file.
\end{itemize}

\index{CRs!control file|(}
\index{configuration files!hwtype.<MAC Address>|(}
\subsection{The CR Control File}
\label{section:cntrlhw}

This section describes the cash register control file:

\begin{Command}{8cm}
hwtype.$<$MAC Address$>$
\end{Command}

The control file is primarily used to set up new cash registers. In this case,
there is no configuration file corresponding to the cash registers
MAC address available. During the boot process of the cash register, the
hardware type and the BIOS version is detected (the POS
hardware manufacturer provides a program for
this function). Using this information, the control file is created, which is
uploaded to the TFTP servers upload directory \textit{/tftpboot/upload}.

\paragraph{\underline{Note:}}
The control file (\textit{hwtype.$<$MAC Address$>$}) is only uploaded to the
TFTP server when no configuration file (\textit{config.$<$MAC Address$>$})
exists. After the configuration file is created, the control file is deleted
from the directory \textit{/tftpboot/upload}.

\begin{Command}{8cm}
\textbf{HWTYPE}=hardware type \\
\textbf{HWBIOS}=bios version  \\
\textbf{CRNAME}=alias name
\end{Command}
\index{CRs!control file|)}
\index{configuration files!hwtype.<MAC Address>|)}

\subsection{Overview of the CR Boot Process}
\label{section:lngovw}

The following section describes the steps that take place when the cash register
is booted with the image determined by its product ID:

\begin{itemize}
\item Via PXE network boot or boot manager (GRUB), the cash register boots the
      initrd (initrd.gz) that it receives from the branch server. If no PXE
      boot is possible, the cash register tries to boot via hard disk, if accessible.
\item Running \textbf{linuxrc} starts the following process:
\end{itemize}

\begin{enumerate}
      \item The required file systems to receive system data are mounted.
            Example: \textbf{proc} file system.
      \item The cash register hardware type (HWTYPE) is detected (the POS
	    hardware manufacturer provides a
            program for this).
      \item The cash register BIOS version (HWBIOS) is detected (the POS
	    hardware manufacturer provides a
            program for this).
      \item Network support is activated. The required kernel module
            is determined from the static table \textbf{/IBMProduct}
            by selecting the entry corresponding
            to the hardware type. If no known hardware type is detected, a
            default list of modules is used and types are tried one after the
            other. The required entries in /IBMproduct  correspond to one
            line each per hardware type in the following format:

            \begin{Command}{8cm}
            hardware type=module name
            \end{Command}

            The module is loaded using \textit{modprobe}. Any dependencies to
            other modules are cleared at that time.
      \item The network interface is set up via DHCP. After the interface has
            been established, the DHCP variables are exported into the file
            \textit{/var/lib/dhcpcd/dhcpcd-eth0.info} and the contents of
            DOMAIN and DNS are used to generate a \textit{/etc/resolv.conf}.
      \item The TFTP server address is acquired. During this step, a check
            is first made to determine whether the host name
            \textbf{tftp.\$DOMAIN} can be resolved. If not, the DHCP
            server is used as the TFTP server.
      \item The configuration file is loaded from the server directory
            \textit{/tftpboot/CR} via TFTP. At this point, the cash register
            expects the file:

            \begin{Command}{8cm}
            config.$<$MAC Address$>$
            \end{Command}

            If this file is not available and cannot be loaded,
                        it means this is a new cash register
            that can be immediately registered.

                        A new cash register is registered
            in two steps:
                        \begin{itemize}
                        \item An optional alias name can be set for the new cash register.
                  During image creation of one of the boot images, you can
                  enable the system alias setting via the \textbf{POSSetAlias}
                  feature module. By default, there is no question for the
                  system alias name.
                        \item A control file is uploaded to the TFTP servers upload
                  directory: \textit{/tftpboot/upload}.

            \begin{Command}{8cm}
            hwtype.$<$MAC Address$>$
            \end{Command}
                        \end{itemize}

                        The contents of the file correspond to the cash register hardware
            type, the BIOS version, and the cash register alias name. Further
            information about the control file can be found in Section
            \ref{section:cntrlhw}.

            After the upload, the cash register branches off into a loop in
            which the following steps are taken:

            \begin{itemize}
            \item the DHCP lease file is renewed (\textit{dhcpcd -n}).
            \item a new attempt is made to load the file
                  config.$<$MAC address$>$ from the TFTP server.
            \item if the file does not exist, there is a 60-second wait
                  period before a new run begins.
            \end{itemize}

            If the configuration file does load, it contains data on
            image, configuration, synchronization, or partition parameters.
                        For more infomation about the file format of the configuration file,
                        refer to Section \ref{section:confmac}.

      \item The PART: line in the configuration is analyzed. If there is a
            PART line in the configuration file, the following analysis
            takes place:
            \begin{itemize}
            \item A check is made using the image version to see whether any
                  local system needs to be updated. If not, local boot process
                  continues immediately. No image download occurs.
            \item No system or update required: cash register hard disk is
                  partitioned.
            \end{itemize}
      \item SYNC: line in configuration file is evaluated. If there is a
            SYNC line, the indicated file is downloaded via TFTP. The only
            value the file contains is the number of seconds of wait time
            required before the multicast download of the cash register image
            starts (sleep). If the file is not present, it proceeds
            immediately.  
            Otherwise, it waits for the indicated time.
      \item Indicated images are downloaded with multicast TFTP.
      \item Checksums checked. Repeat download if necessary.
      \item The CONF: line is evaluated. All the indicated files are loaded
            from the TFTP server and stored in a \textit{/config/} path.
      \item Terminate all the user-land processes based on the boot image
            (dhcpcd -k).
      \item Cash register image is mounted.
      \item The configuration files stored in \textit{/config/...} are copied
            into the mounted cash register image.
      \item The system switches to the mounted cash register image. The root
            file system is converted to the cash register image via
            \textbf{pivot\_root}. All the required configuration files are
            now present, because they had been stored in the cash register image
            or have been downloaded via TFTP.
      \item The boot image is unmounted using an \textbf{exec umount} call.
      \item At termination of linuxrc or the exec call, the kernel initiates
            the \textbf{init} process that starts processing the boot
            scripts as specified \textbf{/etc/inittab}, for example,
            to configure the network interface.
              \end{enumerate}

\newpage
\section{Thin Client Adminstration}

\index{CRs!booting|)}
