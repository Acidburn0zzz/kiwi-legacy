\index{KIWI images!ec2|(}
\chapter{EC2 image - Amazon Elastic Compute Cloud}
\label{chapter:ec2}
\minitoc

% <type ec2accountnr="720220196365" ec2privatekeyfile="/suse/ms/.ec2/pk-cambrian-aws.pem" ec2certfile="/suse/ms/.ec2/cert-cambrian-aws.pem">ec2</type>

% ec2privatekeyfile: AWS user's PEM encoded RSA private key file
% ec2accountnr: The user's EC2 user ID (Note: AWS account number, NOT Access Key ID)
% ec2certfile: AWS user's PEM encoded RSA pubkey certificate file

%What I have done so far:
%========================

%1) created a ec2-ami-tools package and submit it to beta. I hope
%   to get that package on SUSE 11.0 too. This package contains the
%   Amazon tools to create what they call a bundle

%2) add a new image type to kiwi which I call ec2. All you need
%   to do is:
%
%   <type
%      ec2accountnr="AWS account Nr"
%      ec2privatekeyfile="AWS Private Key file"
%      ec2certfile="AWS certificate"
%   >ec2</type>
%
%   All the information about the AWS service is given when
%   you create your Amazon AWS acccount and sign up for the EC2
%   service
%
%   kiwi --create ... -d ... --type ec2
%
%3) The generated image must be transfered over to Amazon which is
%   done by ec2-upload-bundle. I would like to write some documentation
%   about the upload and the activation of an instance as soon as we
%   can use at least one of our SUSE kernels

% ec2-upload-bundle -b <your-s3-bucket> -m /mnt/image.manifest.xml -a <aws-access-key-id> -s <aws-secret-access-key>

%ec2-register <your-s3-bucket>/image.manifest.xml
%IMAGE ami-5bae4b32 

%ec2-run-instances ami-5bae4b32
