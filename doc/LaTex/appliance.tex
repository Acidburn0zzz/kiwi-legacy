\chapter{Creating Appliances with KIWI}
\index{KIWI images!appliance|(}
\label{chapter:appliance}
\minitoc

\section{History}
Traditionally, many computing functions were written as software applications
running on top of a general-purpose operating system. The consumer
(whether home computer user or the IT department of a company) bought a
computer, installed the operating system or configured a pre-installed
operating system, and then installed one or more applications on top of
the operating system. An e-mail server was just an e-mail application
running on top of Linux, Unix, Microsoft Windows, or some other operating
system, on a computer that was not designed specifically for that application.

\section{The KIWI Model}
With KIWI we started to use a different model. Instead of installing
firewall software on top of a general purpose computer/operating system,
the designers/engineers built images that are designed specifically for
the task. These are so called appliances.
When building appliances with KIWI the following proceeding has proven
to work reliably. Nevertheless the following is just a recommendation
and can be adapted to special needs and environments.

\begin{enumerate}
\item First you should choose an appropriate image description template from
      the provided KIWI examples and add/adapt repository and/or
      package names according to the distribution you want to
      build an image for
\item Allow the image to create an in-place git repository to
      allow tracking of non binary changes. This is done by adding
      the following into your \cmd{config.sh}\index{scripts!config.sh} script.

\begin{verbatim}
baseSetupPlainTextGITRepository
\end{verbatim}

\item Prepare the preliminary version of your new appliance by
      calling \cmd{kiwi} \option{prepare}\ldots\ and refer to chapter \ref{chapter:usb}
      (USB image - Live-Stick System) for details.
\item Decide for a testing environment. In my opinion a real
      hardware based test machine which allows to boot from USB
      is a good and fast approach. According to this make sure
      you have a usb type in your \path{config.xml}

\begin{xml}
<type filesystem="ext3 boot="usbboot/suse-...">usb</type>
\end{xml}

\item Create the preliminary live stick image of your new appliance
      by calling \cmd{kiwi}\linebreak[3] \option{create}\ldots After successful creation
      of the image files find an USB stick which is able to store
      your appliance and plug it into a free USB port on your image
      build machine. Use the \cmd{kiwi} \option{bootstick}\ldots call to deploy
      the image on the stick. Refer to chapter \ref{chapter:usb}
      (USB image - Live-Stick System) for details.
\item Plug in the stick on your test machine and boot it.
\item After your test system has successfully booted from stick login
      into your appliance and start to tweak the system according to
      your needs. This includes all actions required to make the
      appliance work as you wish. Before you start take care for
      the following:
      \begin{itemize}
      \item Create an initial package list. This can be done by calling:

\begin{Command}{10cm}
rpm -qa | sort > /tmp/deployPackages
\end{Command}

      \item Check the output of the command \cmd{git status} and
            include everything which is unknown to git and surely
            will not be changed by you and will not become part of the
            image description overlay files to the 
            \path{/.gitignore}\index{configuration files!.gitignore}
            files
      \end{itemize}

      After the initial package list exists and the git repository is
      clean you can start to configure the system. You never should
      install additional software just by installing an unmanaged archive
      or build and install from source. It's very hard to find out
      what binary files had been installed and it's also not architecture
      safe. If there is really no other way for the software to become
      part of the image you should address this issue directly in your
      image description and the \cmd{config.sh} script but not after the initial
      deployment has happened.
\item As soon as your system works as expected your new appliance is
      ready to enter the final stage. At this point you have done several
      changes to the system but they are all tracked and should now become
      part of your image description. To include the changes into your
      image description the following process should be used:
      \begin{itemize}
      \item Check the differences between the currently installed packages
            and the initial deployment list. This can be done by calling:

\begin{Command}{12cm}
rpm -qa | sort > /tmp/appliancePackages
diff -u /tmp/deployPackages /tmp/appliancePackages
\end{Command}

            Add those packages which are labeled with ($+$) to the
            \xmlstarttag{packages \xmlattrval{type}{image}} section of your 
            \path{config.xml}\index{configuration files!config.xml}
            file and remove those packages which has been removed ($-$)
            appropriately. If there are packages which has been removed
            against the will of the package manager make sure you address
            the uninstallation of these packages in your 
            \cmd{config.sh}\index{scripts!config.sh}
            script. If you have installed packages from repositories which
            are not part of your \path{config.xml} file you should also add these
            repositories in order to allow kiwi to install the packages
      \item Check the differences made in the configuration files. This
            can be easily done by calling:

\begin{Command}{8cm}
git diff > /tmp/appliancePatch
\end{Command}

            The created patch should become part of your image description
            and you should make sure the patch is applied when preparing the
            image. According to this the command:

\begin{Command}{8cm}
patch -p0 < appliancePatch
\end{Command}

            needs to be added as part of your \cmd{config.sh} script

      \item Check for new non binary files added. This can be done by
            calling:
            
\begin{Command}{4cm}
git status
\end{Command}

	        All files not under version control so far will be listed by
            the command above. Check the contents of this list make sure
            to add all files which are not created automatically to
            become part of your image description. To do this simply
            clone (copy) these files with respect to the filesystem
            structure as overlay files in your image description \path{root/}
            directory.
      \end{itemize}
\item All your valuable work is now stored in one image description
      and can be re-used in all KIWI supported image types. 
\end{enumerate}

Congratulation!
To make sure the appliance works as expected prepare a new image
tree and create an image from the new tree. If you like you can
deactivate the creation of the git repository which will save you
some space on the filesystem. If this appliance is a server I recommend
to leave the repository because it allows you to keep track of changes
during the live time of this appliance.

\index{KIWI images!appliance|)}
