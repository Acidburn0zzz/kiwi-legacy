\index{KIWI images!xen|(}
\chapter{XEN image - Paravirtual Systems}
\label{chapter:xen}
\minitoc

Xen is a free software virtual machine monitor. It allows several
guest operating systems to be executed on the same computer hardware
at the same time.

A Xen system is structured with the Xen hypervisor as the lowest and
most privileged layer.[1] Above this layer are one or more guest
operating systems, which the hypervisor schedules across the physical
CPUs. The first guest operating system, called in Xen terminology
"domain 0" (dom0), is booted automatically when the hypervisor boots
and given special management privileges and direct access to the physical
hardware. The system administrator logs into dom0 in order to start
any further guest operating systems, called "domain U" (domU) in
Xen terminology.

A xen image is a filesystem based image file which requires the
Xen dom0 running or the project called Xenner which emulates
the capabilities of the domain 0. The image created with kiwi
can only be used together with the xen tools.

\section{Building the suse-xen-guest example}

The example provided with kiwi is based on openSUSE 10.3 and includes
the base pattern. The source repository is expected to be
available below the path /image/CDs/full-10.3-i386. If you don't have
a 10.3 repository there edit the config.xml file of the image description
and change the path accordingly or use the suggested remote repository.

\begin{Command}{12cm}
\begin{verbatim}
cd /usr/share/doc/packages/kiwi/examples
kiwi --prepare ./suse-xen-guest \
     --root /tmp/myxen
\end{verbatim}
\end{Command}

\begin{Command}{12cm}
\begin{verbatim}
kiwi --create /tmp/myxen \
     --type xen -d /tmp/myxen-result
\end{verbatim}
\end{Command}

\section{Using the image}

In order to run a domain U the Xen tool \textbf{xm} needs to be called
in conjunction with the KIWI genereated domain U configuration file

\begin{Command}{13cm}
\begin{verbatim}
xm create -c \
   /tmp/myxen-result/\
   suse-10.3-xen-guest.i686-1.1.2.xenconfig
\end{verbatim}
\end{Command}

\section{Flavours}

With KIWI you can provide the information
required to create a guest configuration as part of the config.xml
file. Additionally you can group special packages which you may only
need in this para virtual environment.

\begin{Command}{13cm}
\begin{verbatim}
<packages type="xen" memory="512" disk="/dev/sda">
   <!-- packages you need in Xen only -->
   <package name="kernel-xen"/>
   <package name="xen"/>
</packages>
\end{verbatim}
\end{Command}

If this information is present KIWI will create a Xen domain U
configuration with 512 MB of RAM and expects the disk at /dev/sda.
Other information like bridged network, power management, etc..
needs to be added manually at the moment. The KIWI Xen
domain U configuration is stored in the file:

\begin{Command}{13cm}
\begin{verbatim}
/tmp/myxen-result/\
    suse-10.3-xen-guest.i686-1.1.2.xenconfig
\end{verbatim}
\end{Command}
