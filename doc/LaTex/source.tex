\index{KIWI Setup of installation sources!instsourcesetup|(}
\chapter{Installation Source}
\label{chapter:instsourcesetup}
\minitoc

Before you start to use any of the examples provided in the following
chapters your build system has to have a valid installation source for
the distribution you are about to create an image for.
By default all examples will connect to the network to find the
installation source. It depends on your network bandwidth how fast
an image creation process is and in almost all cases it is better
to prepare a local installation source first.

\section{Adapt the Example's config.xml}
If you can make
sure you have a local installation source it's important to change
the path attribute inside of the \xmlelement{repository} element of the
appropriate example to point to your local source directory.
A typically default repository element looks like the following:

\begin{verbatim}
<repository type="yast2">
   <!--<source path="/image/CDs/full-11.0-i386"/>-->
   <source path="opensuse://openSUSE:11.0/standard/"/>
</repository>
\end{verbatim}

\section{Create a Local Installation Source}
The following describes how to create a local SUSE installation
source which is stored below the path: \path{/images/CDs}
If you are using the local path as described in this docuement
you only need to flip the given path information inside of
the example \path{config.xml} file.

\begin{enumerate}
\item find your SUSE standard installation CDs or the DVD and
      make them available to the build system. Most linux systems
      auto-mount a previosly inserted media automatically. If this
      is the case you simply can change the directory to the
      auto mounted path below \path{/media}. If your system doesn't mount
      the device automatically you can do this with the following
      command:

\begin{Command}{12cm}
\begin{verbatim}
mount -o loop /dev/<drive-device-name> /mnt
\end{verbatim}
\end{Command}

\item You don't have a DVD but a CD set ? No problem all you need to
      do is copy the contents of \emph{all} CDs into one directory. It's
      absolutly important that you first start with the \emph{last} CD and
      copy the first CD at last. In case of CDs you should have a
      bundly of 4 CDs. Copy them in the order 4 3 2 1

\item Once you have access to the media copy the contents of the
      CDs / DVD to your hard drive. You need at least 4GB free
      space available. The following is intended to create a SUSE
      11.0 installation source:

\begin{Command}{12cm}
\begin{verbatim}
mkdir -p /image/CDs/full-11.0-i386/
cp -a /mnt/* /image/CDs/full-11.0-i386/
\end{verbatim}
\end{Command}

      Remember if you have a CD set start with number 4 first and
      after that unplugg the CD and insert the next one to repeat
      the copy command until all CDs are copied into to /image
\end{enumerate}
