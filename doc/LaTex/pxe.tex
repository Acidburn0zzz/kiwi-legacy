\index{KIWI images!pxe|(}
\chapter{PXE Image---Thin Clients}
\label{chapter:pxe}
\minitoc

A PXE image consists of a boot image and a system image like all other
image types too. But with a PXE image the image files are available
seperately and needs to be copied at specific locations of a network
boot server. PXE is a boot protocol implemented in most BIOS implementations
which makes it so interesting. The protocol sends DHCP requests to assign
an IP address and after that it uses tftp to download kernel and boot
instructions. 

\section{Setting Up the Required Services}

Before you start to build pxe images with kiwi you
should have setup the boot server. The boot server requires the
services atftp and DHCP to run

\subsection{Atftp Server}

In order to setup the atftp server the following steps are required

\begin{enumerate}
\item install the packages atftp and kiwi-pxeboot
\item edit the file \path{/etc/sysconfig/atftpd} and set/modify the following
      variables:
      \begin{itemize}
      \item ATFTPD\_OPTIONS=''\option{daemon} \option{no-multicast}''
      \item ATFTPD\_DIRECTORY=''/srv/tftpboot''
      \end{itemize}
\item run atftpd by calling the command:
      \textbf{rcatftpd start}
\end{enumerate}

\subsection{DHCP Server}

In contrast to the atftp server setup the following DHCP server setup
can only serve as an example. Please note that according to your network
structure the IP addresses, ranges and domain settings needs to be adapted
in order to allow the DHCP server to work within your network. If you
already have a DHCP server running in your network you should make sure
that the filename and next-server information is provided by your server.
The following steps describe how to setup a new DHCP server instance:

\begin{enumerate}
\item install the package dhcp-server
\item create the file \path{/etc/dhcpd.conf} and include the following
      statements:
 
      \begin{verbatim}
option domain-name "example.org";
option domain-name-servers 192.168.100.2;
option broadcast-address 192.168.100.255;
option routers 192.168.100.2;
option subnet-mask 255.255.255.0;
default-lease-time 600;
max-lease-time 7200;
ddns-update-style none; ddns-updates off;
log-facility local7;

subnet 192.168.100.0 netmask 255.255.255.0 {
   filename "pxelinux.0";
   next-server 192.168.100.2;
   range dynamic-bootp 192.168.100.5 192.168.100.20;
}
\end{verbatim}

\item edit the file \path{/etc/sysconfig/dhcpd} and setup the network
      interface the server should listen on:
      \begin{itemize}
      \item DHCPD\_INTERFACE=''eth0''
      \end{itemize}
\item run the dhcp server by calling:
      \cmd{rcdhcpd start}
\end{enumerate}

\section{Building the suse-pxe-client Example}

The example provided with kiwi is based on openSUSE 11.2 and creates an
image for a Wyse VX0 terminal with a 128MB flash card and 512MB of RAM.
The image makes use of the squashfs compressed filesystem and its
root tree is deployed as unified (aufs) based system. 

\begin{Command}{12cm}
\begin{verbatim}
cd /usr/share/doc/packages/kiwi/examples
cd suse-11.2
kiwi --prepare ./suse-pxe-client \
     --root /tmp/mypxe
\end{verbatim}
\end{Command}

\begin{Command}{12cm}
\begin{verbatim}
kiwi --create /tmp/mypxe --type pxe \
     -d /tmp/mypxe-result
\end{verbatim}
\end{Command}

\section{Using the Image}

In order to make use of the image all related image parts needs 
to be copied onto the boot server. According to the example the following
steps needs to be performed:

\begin{enumerate}
\item Change working directory
\begin{verbatim}
cd /tmp/mypxe-result
\end{verbatim}
\item Copy of the boot and kernel image
\begin{verbatim}
cp initrd-netboot-suse-11.2.i686-2.1.1.splash.gz \
   /srv/tftpboot/boot/initrd
cp initrd-netboot-suse-11.2.i686-2.1.1.kernel \
   /srv/tftpboot/boot/linux
\end{verbatim}
\item Copy of the system image and md5 sum
\begin{verbatim}
cp suse-11.2-pxe-client.i686-1.2.8 \
   /srv/tftpboot/image
cp suse-11.2-pxe-client.i686-1.2.8.md5 \
   /srv/tftpboot/image
\end{verbatim}
\item Copy of the image boot configuration\\
      Normally the boot configuration applies to one client which
      means it is required to obtain the MAC address of this client.
      If the boot configuration should be used globaly the KIWI
      generated file can be copied as config.default
\begin{verbatim}
cp suse-11.2-pxe-client.i686-1.2.8.config \
   /srv/tftpboot/KIWI/config.<MAC>
\end{verbatim}
\item Check the PXE configuration file\\
      The PXE configuration controls which kernel and initrd are
      loaded and which kernel parameters are set. When installing
      the kiwi-pxeboot package a default configuration is added.
      To make sure the configuration is valid according to this
      example the file \path{/srv/tftpboot/pxelinux.cfg/default} should
      provide the following information:
\begin{verbatim}
DEFAULT KIWI-Boot

LABEL KIWI-Boot
    kernel boot/linux
    append initrd=boot/initrd vga=0x314
    IPAPPEND 1

LABEL Local-Boot
    localboot 0
\end{verbatim}
\item connect the client to the network and boot
\end{enumerate}

\section{Flavours}

All the different PXE boot based deployment methods are controlled
by the \path{config.<MAC>} (or \path{config.default}) file. When a new client
boots up and there is no client configuration file the new client
is registered by uploading a control file to the tftp server.
The following sections informs about the control and the configuration
file.

\index{configuration files!hwtype.<MAC Address>|(}
\subsection{The PXE Client Control File}
\label{section:cntrlhw}
This section describes the netboot client control file:

\begin{Command}{8cm}
hwtype.$<$MAC Address$>$
\end{Command}

The control file is primarily used to set up new netboot clients. In this
case, there is no configuration file corresponding to the client
MAC address available. Using the MAC address information, the control file
is created, which is uploaded to the TFTP servers upload directory
\path{/var/lib/tftpboot/upload}.

\index{configuration files!config.<MAC Address>|(}
\subsection{The PXE Client Configuration File}
\label{section:confmac}
This section describes the netboot client configuration file:

\begin{Command}{8cm}
    config.$<$MAC Address$>$
\end{Command}

The configuration file contains data about image, configuration,
synchronization, or partition parameters. The configuration file is
loaded from the TFTP server directory \path{/var/lib/tftpboot/KIWI} via TFTP
for previously installed netboot clients. New netboot clients are
immediately registered and a new configuration file with the
corresponding MAC address is created. The standard case for the
deployment of a pxe image is one image file based on a read-write
filesystem which is stored onto a local storage device of the client.
Below, find an example to cover this case.

\begin{verbatim}
DISK=/dev/sda
PART=5;S;x,x;L;/
IMAGE=/dev/sda2;suse-11.2-pxe-client.i686;1.2.8;192.168.100.2;4096
\end{verbatim}

The following format is used:

%% toms: Hmn, underline. Better stay way from it
\begin{Command}{14cm}
    \textbf{IMAGE}=\underline{device;name;version;srvip;bsize;compressed},...,\\
    \textbf{CONF}=\underline{src;dest;srvip;bsize},...,
                             src;dest;srvip;bsize\\
    \textbf{PART}=\underline{size;id;Mount},...,size;id;Mount\\
    \textbf{DISK}=device
\end{Command}

\begin{itemize}
    \item \textbf{IMAGE}\\
        Specifies which image (name) should be loaded with which
        version (version) and to which storage device (device) it
        should be linked, e.g., \path{/dev/ram1} or
        \path{/dev/hda2}. The netboot client partition (device)
        \path{hda2} defines the root file system \path{/} and \path{hda1}
        is used for the swap partition. The numbering of the hard disk
        device should not be confused with the RAM disk device,
        where \path{/dev/ram0} is used for the initial RAM disk and
        can not be used as storage device for the second stage system image.
        SUSE recommends to use the device \path{/dev/ram1} for the
        RAM disk. If the hard drive is used, a corresponding partitioning
        must be performed.
        \begin{itemize}
        \item \textbf{srvip}\\
        Specifies the server IP address for the TFTP download.
        Must always be indicated, except in PART.
        \item \textbf{bsize}\\
        Specifies the block size for the TFTP download. Must always
        be indicated, except in PART. If the block size is too small
        according to the maximum number of data packages (32768),
        \textbf{linuxrc} will automatically calculate a new blocksize for
        the download.
        \item \textbf{compressed}\\
        Specifies if the image file on the TFTP server is compressed and
        handles it accordingly. To specify a compressed image download only
        the keyword \textbf{"'compressed"'} needs to be added. If compressed
        is not specified the standard download workflow is used. \textbf{Note:}
        The download will fail if you specify "'compressed"' and the image isn't
        compressed. It will also fail if you don't specify "'compressed"'
        but the image is compressed. The name of the compressed image has
        to contain the suffix \path{.gz} and needs to be compressed with the
        \cmd{gzip} tool. Using a compressed image will automatically
        \emph{deactivate} the multicast download option of atftp.
    \end{itemize}
    \item \textbf{CONF}\\
        Specifies a comma-separated list of source:target
        configuration files. The source (src) corresponds to the path
        on the TFTP server and is loaded via TFTP. The
        download is made to the file on the netboot client
        indicated by the target (dest).
    \item \textbf{PART}\\
        Specifies the partitioning data. The comma-separated list
        must contain the size (size), the type number (id), and the
        mount point (Mount). The size is measured in MB by default.
        Additionally all size specifications supported by the sfdisk
        program are allowed as well. The type number specifies the ID
        of the partition. Valid ID's are listed via the
        \cmd{sfdisk \option{list-types}} command. The mount specifies the
        directory the partition is mounted to.
        \begin{itemize}
            \item The first element of the list must define the swap
                  partition.
            \item The second element of the list must define the
                  \textbf{root} partition.
            \item The swap partition must not contain a mount point.
                  A lowercase letter \textbf{x} must be set instead.
            \item If a partition should take all the space left on
                  a disk one can set a lower \textbf{x} letter as
                  size specification.
        \end{itemize}
    \item \textbf{DISK}\\
        Specifies the hard disk. Used only with PART and defines
        the device via which the hard disk can be addressed,
        e.g., \path{/dev/hda}.
    \item \textbf{RELOAD\_IMAGE}\\
        If set to a non-empty string, this forces the configured
        image to be loaded from the server even if the image on
        the disk is up-to-date. The primary purpose of this setting
        is to aid debugging. The option is sensible only for disk based
        systems.
    \item \textbf{RELOAD\_CONFIG}\\
        If set to a non-empty string, this forces all config files
        to be loaded from the server. The primary purpose of this setting
        is to aid debugging. The option is sensible only for disk based
        systems.
    \item \textbf{COMBINED\_IMAGE}\\
        If set to an non-empty string, indicates that the both
        image specified needs to be combined into one bootable
        image, whereas the first image defines the read-write
        part and the second image defines the read-only part.
    \item \textbf{KIWI\_INITRD}\\
        Specifies the kiwi initrd to be used for local boot of the
        system. The variables value must be set to the name of the
        initrd file which is used via PXE network boot. If the
        standard tftp setup suggested with the kiwi-pxeboot package
        is used all initrd files resides in the \path{boot/} directory
        below the tftp server path \path{/var/lib/tftpboot}. Because
        the tftpserver do a chroot into the tftp server path you need
        to specify the initrd file as the following example shows:
        \textbf{KIWI\_INITRD=/boot/<name-of-initrd-file>}
    \item \textbf{UNIONFS\_CONFIG}\\
        For netboot and usbboot images there is the possibility to
        use unionfs or aufs as container filesystem in combination
        with a compressed system image. The recommended compressed
        filesystem type for the system image is \textbf{squashfs}.
        In case of a USB stick system the usbboot image will
        automatically setup the unionfs/aufs filesystem. In case of
        a PXE network image the netboot image requires a config.<MAC>
        setup like the following example shows:
        \textbf{UNIONFS\_CONFIG=/dev/sda2,/dev/sda3,aufs}. In this
        example the first device /dev/sda2 represents the read/write
        filesystem and the second device /dev/sda3 represents the
        compressed system image filesystem. The container filesystem
        aufs is then used to cover the read/write layer with the
        read-only device to one read/write filesystem. If a file on
        the read-only device is going to be written the changes inodes
        are part of the read/write filesystem. Please note the device
        specifications in UNIONFS\_CONFIG must correspond with
        the IMAGE and PART information. The following example should
        explain the interconnections:

\begin{verbatim}
IMAGE=/dev/sda3;image/myImage;1.1.1;192.168.1.1;4096
PART=200;S;x,300;L;/,x;L;x
UNIONFS_CONFIG=/dev/sda2,/dev/sda3,aufs
DISK=/dev/sda
\end{verbatim}

        As the second element of the PART list must define the \textbf{root}
        partition it's absolutely important that the first device in
        UNIONFS\_CONFIG references this device as read/write device.
        The second device of UNIONFS\_CONFIG has to reference the given
        IMAGE device name.
    \item \textbf{KIWI\_KERNEL\_OPTIONS}\\
        Specifies additional command line options to be passed to
        the kernel when booting from disk.  For instance, to enable
        a splash screen, you might use 'vga=0x317 splash=silent'.
    \item \textbf{KIWI\_BOOT\_TIMEOUT}\\
        Specifies the number of seconds to wait at the grub boot
        screen when doing a local boot.  The default is 10.
    \item \textbf{NBDROOT}\\
        Mount the system image root filesystem remotely via
        NBD (Network Block Device). This means there is a server
        which exports the root directory of the system image via a
        specified port. The kernel provides the block layer, together with a
        remote port that uses the nbd-server program. For more information
        on how to set up the server, see the nbd-server man pages. The
        kernel on the remote client can set up a special network block
        device named \path{/dev/nb0} using the nbd-client command. After this
        device exists, the mount program is used to mount the root
        filesystem. To allow the KIWI boot image to use that, the following
        information must be provided:
\begin{verbatim}
NBDROOT=NBD.Server.IP.address;\
        NBD-Port-Number;/dev/NBD-Device;\
        NBD-Swap-Port-Number;/dev/NBD-Swap-Device;\
        NBD-Write-Port-Number;/dev/NBD-Write-Device
\end{verbatim}
        The NBD-Device, NBD-Swap-Port-Number, NBD-Swap-Device,
        NBD-Write-Port-Number and NBD-Write-Device
        variables are optional. If the nbd root device is not set,
        the default values (\path{/dev/nb0} , port 2000) applies and if the nbd
        swap device is not set the default values (\path{/dev/nb1}, port 9210)
        applies. The swap space over the network using a network block device
        is only established if the client has less than 48 MB of RAM.
        The optional NBD-Write-Port-Number and NBD-Write-Device specifies
        a write COW location for the root filesystem. aufs is used as
        overlay filesystem in this case.
    \item \textbf{AOEROOT}\\
        Mount the system image root filesystem remotely via AoE
        (ATA over Ethernet). This means there is a server which
        exports a block device representing the the root directory of
        the system image via the AoE subsystem. The block device
        could be a partition of a real or a virtual disk. In order to
        use the AoE subsystem I recommend to install the \textit{aoetools}
        and \textit{vblade} packages from here first:
        \url{http://download.opensuse.org/repositories/system:/aoetools}\\
        Once installed the following example shows how to export
        the local \path{/dev/sdb1} partition via AoE:
\begin{verbatim}
vbladed 0 1 eth0 /dev/sdb1
\end{verbatim}
        Some explanation about this command, each AoE device is identified
        by a couple Major/Minor, with major between 0-65535 and minor
        between 0-255. AoE is based just over Ethernet on the OSI models
        so we need to indicate which ethernet card we'll use. 
        In this example we export \path{/dev/sdb1} with a major value of 0 and
        minor of 1 on the eth0 interface. We are ready to use our partition
        on the network! To be able to use the device kiwi needs the
        information which AoE device contains the root filesystem. In
        our example this is the device \path{/dev/etherd/e0.1}.
        According to this the AOEROOT variable must be set as follows:
\begin{verbatim}
AOEROOT=/dev/etherd/e0.1
\end{verbatim}
        kiwi is now able to mount and use the specified AoE device
        as the remote root filesystem. In case of a compressed read-only
        image with aufs or clicfs the AOEROOT variable can also contain
        a device for the write actions:
\begin{verbatim}
AOEROOT=/dev/etherd/e0.1,/dev/ram1
\end{verbatim}
        Writing to RAM is the default but you also can set another device
        like another aoe location or a local device for writing the data
    \item \textbf{NFSROOT}\\
         Mount the system image root filesystem remotely via NFS
         (Network File System). This means there is a server which
         exports the root filesystem of the network client in such a
         way that the client can mount it read/write. In order to do
         that, the boot image must know the server IP address and
         the path name where the root directory
         exists on this server. The information must be provided as
         in the following example:
\begin{verbatim}
NFSROOT=NFS.Server.IP.address;/path/to/root/tree
\end{verbatim}
         Optionally you can set a UNIONFS\_CONFIG variable which defines
         an aufs based overlay NFS directory or device like:
\begin{verbatim}
UNIONFS_CONFIG=/tmp/kiwi-11.1-cow,nfs,aufs # write to NFS directory
UNIONFS_CONFIG=/dev/ram1,nfs,aufs # write to RAM
\end{verbatim}
         This way you can keep the original root tree clean from any
         modifications
    \item \textbf{KIWI\_INITRD}\\
         Specifies the KIWI initrd to be used for a local boot of the system.
         The value must be set to the name of the initrd file which is
         used via PXE network boot. If the standard TFTP setup suggested
         with the kiwi-pxeboot package is used, all initrd files reside
         in the \path{/srv/tftpboot/boot/} directory. Because the TFTP server
         does a chroot into the TFTP server path, you must specify the
         initrd file as follows:
\begin{verbatim}
KIWI_INITRD=/boot/name-of-initrd-file
\end{verbatim}
    \item \textbf{KIWI\_KERNEL}\\
         Specifies the kernel to be used for a local boot of the system
         The same path rules as described for KIWI\_INITRD applies for
         the kernel setup:
\begin{verbatim}
KIWI_KERNEL=/boot/name-of-kernel-file
\end{verbatim}
	\item \textbf{ERROR\_INTERRUPT}\\
         Specifies a message which is displayed during first deployment.
         Along with the message a shell is provided. This functionality
         should be used to send the user a message if it's clear the
         boot process will fail because the boot environment or something
         else influences the pxe boot process in a bad way.
\end{itemize}

\subsection{User another than tftp as download protocol}

By default all downloads controlled by the kiwi linuxrc code
are performed by an atftp call and therefore uses the tftp
protocol. With PXE the download protocol is fixed and thus you
can't change the way how the kernel and the boot image (initrd)
is downloaded. As soon as linux takes over control the following
download protocols http, https and ftp are supported too.
KIWI makes use of the \cmd{curl} program to support the
additional protocols.

In order to select one of the additional download protocols
the following kernel parameters needs to be setup:

\begin{itemize}
\item \textbf{kiwiserver}\\
      Name or IP address of the server who implements the protocol
\item \textbf{kiwiservertype}\\
      Name of the download protocol which could be one of http, https 
      or ftp
\end{itemize}

To setup this parameters edit the file
\path{/srv/tftpboot/pxelinux.cfg/default} on your PXE boot server
and change the \texttt{append} line accordingly. Please note all
downloads except for kernel and initrd are now controlled by the given
server and protocol. You need to make sure that this server provides
the same directory and file structure as initially provided by
the kiwi-pxeboot package.

\subsection{RAM only image}

If there is no local storage and no remote root mount setup
the image can be stored into the main memory of the client.
Please be aware that there should be still enough RAM space
available for the operating system after the image has been
deployed into RAM. Below, find an example:

\begin{itemize}
\item Use a read-write filesystem in \path{config.xml}, for example\\
      \textbf{filesystem=''ext3''}
\item Create \path{config.<MAC>}

\begin{Command}{12cm}
\begin{verbatim}
IMAGE=/dev/ram1;suse-11.2-pxe-client.i686;\
      1.2.8;192.168.100.2;4096
\end{verbatim}
\end{Command}
\end{itemize}

\subsection{Union Image}

As used in the suse-pxe-client example it is possible to make use of the
aufs or unionfs overlay filesystems to combine two filesystems
into one. In case of thin clients there is often the need for
a compressed filesystem due to space limitations. Unfortunately
all common compressed filesystems provides only read-only access.
Combining a read-only filesystem with a read-write filesystem
is a solution for this problem. In order to use a compressed
root filesystem make sure your config.xml's filesystem attribute
contains either squashfs or clicfs. Below, find an example:

\begin{Command}{12cm}
\begin{verbatim}
DISK=/dev/sda
PART=5;S;x,62;L;/,x;L;x,
IMAGE=/dev/sda2;suse-11.2-pxe-client.i386;\
      1.2.8;192.168.100.2;4096
UNIONFS_CONFIG=/dev/sda3,/dev/sda2,aufs
KIWI_INITRD=/boot/initrd
\end{verbatim}
\end{Command}

\subsection{Split Image}

As an alternative to the UNIONFS\_CONFIG method it is also
possible to create a split image and combine the two portions
with the COMBINED\_IMAGE method. This allows to use different
filesystems without the need for an overlay filesystem to combine
them together. Below find an example:

\begin{itemize}
\item add a split type in config.xml, for example\\
      \textbf{<type image="split" fsreadonly=''squashfs'' fsreadwrite=''ext3''
      boot=''netboot/suse-11.2''/>}
\item add a split section inside the type to describe the temporary
      and persistent parts. For example:
\begin{verbatim}
<split>
   <temporary>
      <!-- allow RAM read/write access to: -->
      <file name="/mnt"/>
      <file name="/mnt/*"/>
   </temporary>
   <persistent>
      <!-- allow DISK read/write access to: -->
      <file name="/var"/>
      <file name="/var/*"/>
      <file name="/boot"/>
      <file name="/boot/*"/>
      <file name="/etc"/>
      <file name="/etc/*"/>
      <file name="/home"/>
      <file name="/home/*"/>
   </persistent>
</split>
\end{verbatim}
\item sample config.<MAC>
     
\begin{Command}{14cm}
\begin{verbatim}
IMAGE=/dev/sda2;suse-11.2-pxe-client.i686;\
      1.2.8;192.168.100.2;4096,\
      /dev/sda3;suse-11.2-pxe-client-read-write.i686;\
      1.2.8;192.168.100.2;4096
PART=200;S;x,500;L;/,x;L;
DISK=/dev/sda
COMBINED_IMAGE=yes
KIWI_INITRD=/boot/initrd
\end{verbatim}
\end{Command}
\end{itemize}

\subsection{Root Tree Over NFS}

Instead of installing the image onto a local storage device of
the client it is also possible to let the client mount the root
tree via an NFS remote mount. Below find an example:

\begin{itemize}
\item Export the kiwi prepared tree via NFS
\item sample config.<MAC>

\begin{Command}{12cm}
\begin{verbatim}
NFSROOT=192.168.100.7;/tmp/kiwi.nfsroot
\end{verbatim}
\end{Command}
\end{itemize}

\subsection{Root Tree Over NBD}

As an alternative for root over NFS it is also possible to
let the client mount the root tree via a special network block
device. Below find an example:

\begin{itemize}
\item Use nbd-server to export the kiwi prepared tree
\item sample config.<MAC>

\begin{Command}{12cm}
\begin{verbatim}
NBDROOT=192.168.100.7;2000;/dev/nbd0
\end{verbatim}
\end{Command}
\end{itemize}

\subsection{Root Tree Over AoE}

As an alternative for root over NBD it is also possible to
let the client mount the root device via a special ATA over
Ethernet network block device. Below find an example:

\begin{itemize}
\item Use the vbladed command to bind a block device to an
      ethernet interface. The block device can be a disk
      partition or a loop device (losetup) but not a directory 
      like with NBD
\item sample config.<MAC>

\begin{Command}{12cm}
\begin{verbatim}
AOEROOT=/dev/etherd/e0.1
\end{verbatim}
\end{Command}

      This would require the command
      \cmd{vbladed 0 1 eth0 blockdevice}
      to be called first
\end{itemize}

