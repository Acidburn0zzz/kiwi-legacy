\index{KIWI images!oem|(}
\chapter{OEM image - Preload Systems}
\label{chapter:oem}
\minitoc

An oem image is a virtual disk image representing all partitions
and bootloader information like it exists on a real disk. The image
format is the same compared to the VMX image type. All flavours
explained in the VMX chapter also applies to the OEM type.

The original idea of an oem image is to provide this virtual disk data
to OEM vendors which now are able to deploy the system independently onto
their storage media. The deployment can happen from any OS including
Windows if a tool to dump data on a disk device exists. The oem image
type is also used to deploy images on USB sticks because in principal
it is the same workflow.

\section{Building the suse-oem-preload example}

The example provided with kiwi is based on openSUSE 10.3 and includes
the patterns default plus x11. The image type is a split type whereas
the read-write filesystem is ext3 and the read-only filesystem is
squashfs. The additional format attribute also creates an installable
ISO image for deploying the image from CD. The source repository is expected
to be available below the path /image/CDs/full-10.3-i386. If you don't have
a 10.3 repository there edit the config.xml file of the image description
and change the path accordingly or use the suggested remote repository.

\begin{Command}{12cm}
\begin{verbatim}
cd /usr/share/doc/packages/kiwi/examples
kiwi --prepare ./suse-oem-preload \
     --root /tmp/myoem
\end{verbatim}
\end{Command}

\begin{Command}{12cm}
\begin{verbatim}
kiwi --create /tmp/myoem --type split \
     -d /tmp/myoem-result
\end{verbatim}
\end{Command}

\section{Using the image}

Testing the oem virtual disk can be done with a virtualisation software
like QEMU or VMware. The virtual disk is represented by the .raw extension
whereas the .iso extension represents the installation disk for this oem
image. The installation disk should be tested on a bare test system
For the .raw test just call:

\begin{Command}{12cm}
\begin{verbatim}
cd /tmp/myoem-result
qemu suse-10.3-oem-preload.i686-1.1.2.raw \
     -m 512
\end{verbatim}
\end{Command}

or dump the image on a test hard disk and select it as boot device
in the BIOS:

\begin{Command}{12cm}
\begin{verbatim}
cd /tmp/myoem-result
dd if=suse-10.3-oem-preload.i686-1.1.2.raw \
   of=/dev/<device> bs=32k
\end{verbatim}
\end{Command}

Please note if you test an oem image the virtual disk geometry of the
image is the same as the disk geometry inside the host system. According
to this the oem boot workflow will skip the re-partitioning which is
performed if there would be a real disk

\section{Flavours}

An interesting part of an oem image is that it can be turned into
an installation image too. This means it is possible to create an
installation CD / DVD or USB stick which deploysthe oem based
image onto the selected storage device. The installation process
is a simply dd of the image onto the selected device so don't
expect any user interaction or GUI here to pop up. KIWI supports
two types of installation media:

\begin{itemize}
\item <type ... format=''iso''>...</type>\\
      Creates a .iso file which can be burned in CD or DVD. This
      represents an installation CD
\item <type ... format=''usb''>...</type>\\
      Creates a .raw.install file which can be dumped (dd) on a
      USB stick. This represents an installation Stick
\end{itemize}

\subsection{Influencing the oem partitioning}

By default the oemboot process will create/modify a swap, /home and /
partition. When providing the file \textbf{/config.oempartition} as part
of the oemboot boot image one can influence this partitioning process.
The format of the file is a simple key=value format. Following variables
can be set:

\begin{itemize}
\item \textbf{OEM\_REBOOT=1}\\
      This allows to reboot the oem system after initial deployment.
      Don't specify OEM\_REBOOT at all if you want to boot directly into
      the system
\item \textbf{OEM\_SWAPSIZE=<number in MB>}\\
      Set the size of the swap partition
\item \textbf{OEM\_SYSTEMSIZE=<number in MB>}\\
      Set the size of the / partition
\item \textbf{OEM\_WITHOUTHOME=1}\\
      Specify if a home partition should be create. Don't specify
      OEM\_WITHOUTHOME at all if you want a /home partition
\item \textbf{OEM\_WITHOUTSWAP=1}\\
      Specify if a spaw partition should be create. Don't specify
      OEM\_WITHOUTSWAP at all if you want a swap partition
\item \textbf{OEM\_BOOT\_TITLE=text}\\
      By default the string \textbf{OEM} will be appended to the
      boot manager menu when KIWI creates the grub configuration during
      first deployment. The variable OEM\_BOOT\_TITLE allows to set
      a custom name which is used instead of OEM.
\end{itemize}

If you plan to use a custom partition setup you also should consider
to create a copy of the original oemboot boot image description. The following
workflow should be used to integrate a custom config.oempartition:

\begin{enumerate}
\item Clone the used boot image description
\begin{verbatim}
cp -a /usr/share/kiwi/image/oemboot/suse-... \
      /usr/share/kiwi/image/oemboot/suse-...-myoem
\end{verbatim}
\item Copy the config.oempartition file into the boot image description 
\begin{verbatim}
cp config.oempartition \
   /usr/share/kiwi/image/oemboot/suse-...-myoem/root
\end{verbatim}
\item Sign your new boot image description
\begin{verbatim}
kiwi --createhash \
     /usr/share/kiwi/image/oemboot/suse-...-myoem
\end{verbatim}
\item Make sure your system image use the new boot image
\begin{verbatim}
<type ... boot="oemboot/suse-...-myoem">...</type>
\end{verbatim}
\end{enumerate}

